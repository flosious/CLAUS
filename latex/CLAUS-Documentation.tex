%     Copyright (C) 2020 Florian Bärwolf
% 
%     This program is free software: you can redistribute it and/or modify
%     it under the terms of the GNU General Public License as published by
%     the Free Software Foundation, either version 3 of the License, or
%     (at your option) any later version.
% 
%     This program is distributed in the hope that it will be useful,
%     but WITHOUT ANY WARRANTY; without even the implied warranty of
%     MERCHANTABILITY or FITNESS FOR A PARTICULAR PURPOSE.  See the
%     GNU General Public License for more details.
% 
%     You should have received a copy of the GNU General Public License
%     along with this program.  If not, see <https://www.gnu.org/licenses/>.

\documentclass[a4paper,10pt]{article}
\usepackage{a4wide}
\usepackage[utf8]{inputenc}
% \usepackage[ngerman]{babel}
\usepackage[english]{babel}
\usepackage{hyperref}
\hypersetup{
    colorlinks,
    citecolor=black,
    filecolor=black,
    linkcolor=black,
    urlcolor=black
}


\title{CLAUS \\ CLever AUtomated Scientist}
\author{Florian Bärwolf  \\
	baerwolf@ihp-microelectronics.com \\
% 	\and 
% 	The Other Dude \\
% 	His Company / University \\
	}

\date{\today}
% Hint: \title{what ever}, \author{who care} and \date{when ever} could stand 
% before or after the \begin{document} command 
% BUT the \maketitle command MUST come AFTER the \begin{document} command! 
\begin{document}

\maketitle

\tableofcontents
\newpage

% \begin{abstract}
% Short introduction to subject of the paper \ldots 
% \end{abstract}

\section{Introduction}
This is a documentation for CLAUS. CLAUS is a program, which automates the calculation process of raw data measurement files (e.g. intensities) into sample property files (e.g. concentrations). It is developed under Linux in objectorientated C++11, is 100\% STL compatible and can therefor be easily (cross) compiled to various platforms and operating systems. It is a very simple command line program with no GUI.

\section{Version}
The current version is ``2020-06-21\_7 ALPHA''

\section{Motivation}
Calculation processes became more and more complex over the past years. There are many places for a scientist to make errors: swapping numbers, using wrong dose or concentration or sample \ldots \\
Scientists or operators might use different calculation methods or models, which lead to different results. \\
The main motivation is to decrease these errors, to unify and clarify calculation methods within one software for all measurements and calculation methods. \\
The reason for using C++: there are many established 3rd party libraries like GSL or sqlite, it is close to the actual hardware, so it is pretty fast and reliable (about 100 times faster compared to matlab), it is free for everybody and can be used on countless platforms and operating systems and I personally already knew it.

\section{Definitions}
\begin{description}
\item[sample\label{sample}]{\ldots contains all data which are independent from measurement tool parameters. E.g. material properties like concentration over depth, substrate, layers, ... . Samples are distinguished from eachother by their lot, waferid, monitor and chip X/Y from the file name.}

\item[isotope\label{isotope}]{\ldots e.g. ``74Ge2`` is defined by the chemical element nucleon number and its symbol. The number of atoms is optional.}

\item[cluster\label{cluster}]{\ldots e.g. ``74Ge2 31P`` is made out of a certain amount of isotopes. A cluster containing only exactly 1 isotope is an isotope.}

\item[measurement\_group\label{measurement_group}]{\ldots contains all measurements measured by the same measurement tool parameters, their common calculation parameters (like gRSF, gSR, ...). ;easurement\_groups are distinguished from eachother by their olcdbid, groupid, measurement tool and actual measured clusters/elements. That means all measurements within a measurement group must have data to exactly the same clusters.}

\item[measurement\label{measurement}]{\ldots contains all data from the clusters of the primary file and its secondary measurement files (e.g. craters). It is a mixture of sample properties and measurement tool parameters. A measurement is default defined by its file name parts: olcdbid, lot, wafer, chip (if set), monitor(if set), groupid, repetition}

\item[measurement\_tool\label{measurement_tool}]{\ldots is determined by the data parser from the input files (e.g. dsims, tofsims, xps, ...) and possible measurement settings like sputter element, energy, polarity}

\item[calculation\_manager\label{calculation_manager}]{\ldots allows calculation over all measurement\_groups. not implemented yet.}

\item[secondary\_measurement\label{secondary_measurement}]{\ldots is a measurement of something of the first measurement created/destroyed. e.g. a crater measurement generated by SIMS is a secondary measurement}

\end{description}

\section{Functionalities}
This program was designed to be CPU time efficient from the ground. Many calculation results will be saved in the hope to be useful at a different point in run time. Sacrificing memory over CPU time. Execution time for about 1900 files was 44 sec on intel i5@4.4GHz, while memory usage was peaking at 115MB RAM.
\subsection{file name parsing}
The correct input file name format is \textbf{\textit{very}} important. The program \textbf{\textit{needs}} these information to calculate correctly. \\
File name is splitted in to parts distinguished by a delimiter ``\_''. The file name parts are:
\begin{itemize} 
 \item necessary: olcdbid, groupid, file typ ending, lot/samplename
 \item optional: lotsplit, monitor, chipX/Y, crater depths, sputter energy, sputter element, sputter polarity, repetition, lot, wafer number
\end{itemize}
The algorithm tries to parse these parts as a regular expression in a specific order:
\begin{enumerate}
 \item olcdbid: ... will be parsed as an integer so ``0001'' is same number like ``1''
 \item wafer: ... will be parsed as integer
 \item lot: ...
 \item chip X/Y
 \item monitor
 \item groupid
 \item sputter energy, polarity, element
 \item crater depths
 \item repetition: a char to distinguish multiple measurements of exactky the same sample in exactly the same measurement group. It is added directly after the group. 
 \item not parseable
\end{enumerate}

some filename examples:
\begin{itemize}
 \item 51880\_ECY011\_w12\_c6-4\_g4.TXT            
 \item 51880\_ECY011\_w14\_X06Y05\_g4.tofsims.txt
 \item 51880\_ECY011\_w12\_X06Y04\_g4.tofsims.txt  
 \item 51880\_MHZ115\_w11\_g4.tofsims.txt
 \item 51880\_ECY011\_w13\_c6-4\_g4.TXT            
 \item 51880\_MHZ115\_w11\_g4.TXT
 \item 51880\_ECY011\_w13\_X06Y04\_g4.tofsims.txt  
 \item 51880\_SJZ292\_w01\_g4.tofsims.txt
 \item 51880\_ECY011\_w14\_c6-5\_g4.TXT
 \item 51880\_SJZ292\_w01\_g4.TXT
 \item 52022\_C3659\_450VO+\_g2.dp\_rpc\_asc
\end{itemize}


\subsubsection{Example 1}
file name = 51987\_A\_500VCs-\_g1q\_325nm\_3255A.TXT
\begin{itemize}
 \item olcdbid = 51987
 \item groupid = 1
 \item file typ ending = TXT
 \item sputter polarity = -
 \item sputter element = Cs
 \item sputter energy = 500 [eV]
 \item repetition = q
 \item crater depths = 325 [nm], 325.5 [nm]
 \item not parseable = A
 \item chipX/Y is empty
 \item lotsplit is empty
 \item lot is empty
 \item wafer number is empty
 \item monitor is empty
\end{itemize}

\subsubsection{Example 2}
file name = 51523\_ZA22\_w23\_chip-03-04\_450VO+\_8q.dp\_rpc\_asc
\begin{itemize}
 \item olcdbid = 51523
 \item groupid = 8
 \item file typ ending = dp\_rpc\_asc
 \item sputter polarity = +
 \item sputter element = O
 \item sputter energy = 450 [eV]
 \item repetition = q
 \item chipX/Y: X=3, Y=4
 \item lot = ZA22
 \item wafer = 23
 \item crater depths is empty
 \item lotsplit is empty
 \item monitor is empty
 \item not parseable is empty
\end{itemize}

\subsubsection{Example 3}
file name = 52022\_C3659\_450VO+\_g2w.dp\_rpc\_asc
\begin{itemize}
 \item olcdbid = 52022
 \item groupid = 2
 \item file typ ending = dp\_rpc\_asc
 \item sputter polarity = +
 \item sputter element = O
 \item sputter energy = 450 [eV]
 \item repetition = w
 \item chipX/Y is empty
 \item lot = C3659
 \item wafer is empty
 \item crater depths is empty
 \item lotsplit is empty
 \item monitor is empty
 \item not parseable is empty
\end{itemize}

\subsection{file filtering}
Filters files with a specified file format from a config file or ignores them. Only the wildcard ``*'' is allowed, which is very useful in windows to apply multiple files in one action to CLAUS.\\
You can specifiy the following commands in a config file:\\
\begin{itemize}
 \item ``ignore\_file\_type\_endings`` will ignore the endings of the input files and tries to parse the contents no matter what
 \item ''ignore\_filename\_substrings`` will ignore all files containing this string within their file names
 \item ''ignore\_filename`` will ignore this specific file
 \item ''use\_directory\_files\_list`` if set to 1, will load all files in all directories of each input file and tries to use them for calculation
\end{itemize}


\subsection{file content parsing}
After filtering the input files and parsing the file name, the algorithm will try to parse the contents of the files with several parser methods origining from the different measurement tool exports
\begin{itemize}
 \item dsims: dp\_rpc\_asc
 \item tofsims: txt
 \item images: jpg, jpeg (just parsing file name, no contents yet)
 \item dektak32: txt
 \item xps: csv
\end{itemize}

\subsection{calculation}
Standard SIMS calculation techniques are utilized, as well as jiangs\cite{jiang} protocol.

\subsection{database}
A sqlite3 file, which contains the reference parameters. At the moment this database will not be automatically populated.

\subsection{export}
To export the measurement groups, measurements and clusters. Very rudimentary class at the moment.

\subsection{plotting}
Utilizing gnuplot, this is only possible under Linux at the moment.

\section{calculation methods}
If the matrix of all measurements within the measurement group consists of exactly one element, e.g. 100at\% Si, than the following methods will be applied to get the concentration and sputter depth.

\subsection{sputter depth}
\begin{enumerate}
 \item skalar: Sputter depth = sputter time * sputter rate 
 \item vector: Sputter depth = integrate(sputter time) d(sputter rate) for each point
\end{enumerate}

\subsection{total sputter depth}
If there were crater depths in the file name, then their mean will be used. 
If there are no crater depths in the file name, then the mean of the depths of the fitted line profiles will be used.

\subsection{sputter rate}
The equilibrium position will NOT be used.
\begin{enumerate}
 \item sputter rate =  total sputter depth / total sputter time 
 \item sputter time of fitted (polynom, 17th grade) maximum concentration/intensity will be used to calculate sputter rate from database maximum depth.
 \item mean of all sputter rates from other measurements in the group
\end{enumerate}

\subsection{RSF}
SF * reference intensity

\subsection{SF}
\begin{enumerate}
 \item For reference samples: if the last value of the implanted isotope/cluster is lower than 5\% compared to the maximum value of the implant, then integrate the whole intensity profile starting from the equilibrium position
 \item For reference samples: maximum intensity / maximum concentration from database
 \item mean of all RSFs for this cluster from other measurements in the group / reference intensity: SF= mean(RSFs) * reference intensity point by point
\end{enumerate}

\subsection{concentration}
concentration = SF * intensity


\subsection{reference intensity}
Is the median of the sum of the intensities of all reference clusters.

\subsection{equilibrium position}
The algorithm to determine the equilibrium position for a measurement is quite compilcated. If there is a cluster with a minimum position in the vicinity of the surface, than this will be used as equlibirum position.

\subsection{jiangs protocol}
Will only be applied to substrates consisting out of exactly 2 elements. If the measured reference samples spread over the whole substrate range (0-100at\%) then jiangs protocol can not be applied. Keep the relative concentrations of all samples below 50at\% or you should know what you are doing.\\
Matrix concentration and SRs are interpolated linearly. RSFs are interpolated by highest polynom grade possible (e.g. cubic for 4 reference samples).


\subsection{planned}
\begin{itemize}
 \item bug fixes
 \item more bug fixes
 \item ... fix bugs
\end{itemize}


\section{How to use CLAUS?}
Define a config file from the template. Be sure, that the filenames of your exported files are correctly formatted, then simply drag and drop them with one or more config files on to the executable / binary.
Commandline usage: \\
linux: ''./claus config.conf samplefile1 samplefile2 *multiplefiles*`` \\
windows: ''claus.exe config.conf samplefile1 samplefile2 *multiplefiles*`` \\
When you have multiple measurements or secondary measurements of the same sample, you can try to give them to the program at the same time. CLAUS will try to use all of the given information from different tools to calculate the results or minimize the uncertanties.  \\
\textbf{If you mix up sample names, results can and most likely will be wrong.}

\section{config file parameters}
Necessary:
\begin{itemize}
 \item ''pse``: The file name with path to the periodic table of elements
\end{itemize}
Optional:
\begin{itemize}
 \item measurement\_definition = olcdbid n lot n wafer n monitor n chip n groupid n repetition n lot\_split defines a measurement.
%  \item ''measurement\_group\_definition``= ''settings`` + ''olcdbid`` + ''groupid`` + ''tool`` defines a measurement_group.
%  \item ''threads``: number of CPU thread the program should use. Not used at the moment.
%  \item ''export\_filename``: the name of the exported files. Place holders can be used, like: {olcdb}, {wafer}, {lot} ,{lot_split}, {chip}, {monitor}, {tool}, {energy}
%  \item ''export\_location``: the path where to save all exported files in
%  \item ''plots\_location``: the path where to save all plots, generated
%  \item ''calc\_location``: the path where to save all calculation results
%  \item ''ignore\_filename\_substrings``: ignores all input files, which contain this string
%  \item ''print_errors``: default is 0, but if set to 1 a list of errors will be printed out.
%  \item ''ignore\_file\_type\_endings``: default is 0, but if set to 1 the program will try to parse the contents no matter what.
%  \item ''db\_location``: the path and filename of the database. If nothing is set, the database will be put in the applications directory and named database.sqlite3
%  \item ''sputter\_time\_resolution'': defines the numerical resolution of the exported data. If nothing is set, the numerical resolution will be similar to physical resolution
%  \item ``sputter\_depth\_resolution'': see sputter\_time\_resolution
%  \item ``replace'': {this} = {with this} replaces string in the loaded files, before parsing them
%  \item ``file\_name\_delimiter'': the delimiter for the file name parts: default is ``\_'' .
%  \item ``use\_directory\_files\_list'': will load all files from all directories of each input file
%  \item ``use\_wildcards\_in\_filenames'' if set to 1, it will activate the ``*'' shell operator. Very useful in windows.
\end{itemize}

\section{external libraries}
The GNU Scientific Library (GSL) is used for statistics analysis. The sqlite3 library is utilized to communicate with the database. The program GNUplot can be used under Linux to directly plot some graphs.

\section{license}
Copyright (C) 2020 Florian Bärwolf \\
baerwolf@ihp-microelectronics.com\\
\\
This program is free software: you can redistribute it and/or modify it under the terms of the GNU General Public License as published by the Free Software Foundation, either version 3 of the License, or (at your option) any later version. \\
\\
This program is distributed in the hope that it will be useful, but WITHOUT ANY WARRANTY; without even the implied warranty of MERCHANTABILITY or FITNESS FOR A PARTICULAR PURPOSE.  See the GNU General Public License for more details.\\
\\
You should have received a copy of the GNU General Public License along with this program.  If not, see https://www.gnu.org/licenses/.

\section{History}
Developing CLAUS started several years ago somewhere in 2017, but it was named different back in the days and cosisted of multiple different programs. Many other programs preceeded this one, until it was planned from the ground new. Parts of this program date back to the year 2009. It was structurally planned and designed from the ground up for its purpose.


% \begin{description}
% \item[article\label{article}]{Article is \ldots}
% \item[book\label{book}]{The book class \ldots}
% \item[report\label{report}]{Report gives you \ldots}
% \item[letter\label{letter}]{If you want to write a letter.}
% \end{description}


% \section{Conclusions}\label{conclusions}
% There is no longer \LaTeX{} example which was written by \cite{doe}.


\begin{thebibliography}{9}
% \bibitem{doe} \emph{First and last \LaTeX{} example.},John Doe 50 B.C. 
\bibitem{jiang} \emph{Jiang Z. X. et al 2006 Appl. Surf. Sci. 252 7262} 
\bibitem{baerwolf} \emph{Bärwolf et al Semicond. Sci. Technol. 34 (2019) 014005 (6pp)} 
\end{thebibliography}

\end{document}
